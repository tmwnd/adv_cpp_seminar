\subsubsection{Views}

\begin{frame}{Views}
    \begin{itemize}
        \item<1-> Views leben in \mintinline{c++}{std::ranges::views}
        \item<2-> Es gibt den alias \mintinline{c++}{std::views}
        \item<3-> Was kennzeichnet Views aus?
            \begin{itemize}
                \item<4-> lazy evaluation
                \item<5-> besitzen Inhalt nicht, referenzieren nur
                \item<6-> constant time copy and move
                \item<7-> einfach und effizient kombinierbar
            \end{itemize}
    \end{itemize}
\end{frame}

%Bemerkung: using directives

\begin{frame}{Views - Beispiel}
    \textbf{Aufgabe:}\\
    Berechnen Sie für alle Pokémon des Typs Feuer den BMI

    \vspace{2.5em}

    \begin{center}
        $\text{bmi} = \frac{\text{gewicht}}{\text{groesse}^2}$
    \end{center}
\end{frame}

\begin{frame}[fragile]{Klassisch}
    \begin{minted}[fontsize=\footnotesize]{c++}
std::vector<Pokemon> get_pokemon() {...};
auto pokemon = get_pokemon();
std::vector<int> bmis{};

for (const Pokemon& p : vec) {
    if (p.primaer_typ == "Feuer" || p.sekundaer_typ = "Feuer") {
        int bmi = p.gewicht() / (p.groesse * p.groesse);
        bmis.push_back(bmi);
    }
}
    \end{minted}
\end{frame}

\begin{frame}[fragile]{alte STL}
    \begin{minted}{c++}
std::vector<Pokemon> get_pokemon() {...};
auto pokemon = get_pokemon();
std::vector<double> bmis;
auto hat_typ_feuer = [](const Pokemon &p) {
    return p.primaer_typ == "Feuer"
        || p.sekundaer_typ = "Feuer";
};
auto bmi = [](const Pokemon &p) {
    return (p.gewicht / (p.groesse * p.groesse));
};

    \end{minted}
\end{frame}

\begin{frame}[fragile]{alte STL}
    \begin{minted}{c++}
std::copy_if(
    pokemon.begin(), pokemon.end(),
    std::back_inserter(feuer_pokemon),
    hat_typ_feuer);
std::transform(
    feuer_pokemon.begin(), feuer_pokemon.end(),
    std::back_inserter(bmi),
    bmi);
    \end{minted}
\end{frame}

\begin{frame}[fragile]{Mit Ranges}
    \begin{minted}{c++}
std::vector<Pokemon> get_pokemon() {...};
auto pokemon = get_pokemon();
auto hat_typ_feuer = [](const Pokemon &p) {
    return p.primaer_typ == "Feuer"
        || p.sekundaer_typ = "Feuer";
};
auto bmi = [](const Pokemon &p) {
    return (p.gewicht / (p.groesse * p.groesse));
};

auto view = transform(filter(vec, hat_typ_feuer), bmi);

\end{minted}
\end{frame}

\begin{frame}[fragile]{Mit Ranges und Pipes}
    \begin{minted}{c++}
std::vector<Pokemon> get_pokemon() {...};
auto pokemon = get_pokemon();
auto hat_typ_feuer = [](const Pokemon &p) {
    return p.primaer_typ == "Feuer"
        || p.sekundaer_typ = "Feuer";
};
auto bmi = [](const Pokemon &p) {
    return (p.gewicht / (p.groesse * p.groesse));
};

auto view = vec | filter(hat_typ_feuer) | transform(bmi);

    \end{minted}
\end{frame}

\begin{frame}[fragile]{Mit Ranges und Pipes}
    \begin{minted}{c++}
std::vector<Pokemon> get_pokemon() {...};
auto pokemon = get_pokemon();
auto view = vec
| filter([](const Pokemon &p) {
    return p.primaer_typ == "Feuer"
        || p.sekundaer_typ = "Feuer";
})
| transform([](const Pokemon &p) {
    return (p.gewicht / (p.groesse * p.groesse));
});
    \end{minted}
\end{frame}

\begin{frame}[fragile]{Dangling Iterator}
    \begin{overlayarea}{\linewidth}{6cm}
        \begin{minted}[]{c++}
std::vector<Pokemon> get_pokemon() {...};

auto pokemon = get_pokemon();
auto it = min_element(
    pokemon, {},
    &Pokemon::groesse
);

std::cout << *it << std::enl;
        \end{minted}
    \end{overlayarea}
\end{frame}

\begin{frame}[fragile]{Dangling Iterator}
    \begin{overlayarea}{\linewidth}{6cm}
        \begin{minted}[]{c++}
std::vector<Pokemon> get_pokemon() {...};

auto it = min_element(
    get_pokemon(), {},
    &Pokemon::groesse
);
std::cout << *it << std::enl;
    \end{minted}
        \begin{onlyenv}<2->
            \begin{minted}{c++}
//error: no match for ‘operator*’
//(operand type is ‘std::ranges::dangling’)
            \end{minted}
        \end{onlyenv}
    \end{overlayarea}
\end{frame}

\begin{frame}[fragile]{Dangling View}
    \begin{overlayarea}{\linewidth}{6cm}
        \begin{minted}{c++}
std::vector<Pokemon> get_pokemon() {...};
auto hat_typ_feuer = [](const Pokemon &p) { 
    return p.primaer_typ == "Feuer"
        || p.sekundaer_typ = "Feuer";
};

auto pokemon = get_pokemon();
auto view = views::filter(pokemon, hat_typ_feuer);
for(const Pokemon& p : view) {
    std::cout << p << std::endl;
}
    \end{minted}
        \begin{onlyenv}<-0>
            \begin{minted}{c++}
//note: candidate: ‘template<class _Range, class _Pred>
//requires (viewable_range<_Range>)
            \end{minted}
        \end{onlyenv}
    \end{overlayarea}
\end{frame}

\begin{frame}[fragile]{Dangling View}
    \begin{overlayarea}{\linewidth}{6cm}
        \begin{minted}{c++}
std::vector<Pokemon> get_pokemon() {...};
auto hat_typ_feuer = [](const Pokemon &p) { 
    return p.primaer_typ == "Feuer"
        || p.sekundaer_typ = "Feuer";
};

auto view = views::filter(get_pokemon(), hat_typ_feuer);
for(const Pokemon& p : view)
    std::cout << p << std::endl;
    \end{minted}
        \begin{onlyenv}<2->
            \begin{minted}{c++}
//note: candidate: ‘template<class _Range, class _Pred>
//requires (viewable_range<_Range>)
            \end{minted}
        \end{onlyenv}
    \end{overlayarea}
\end{frame}


\begin{frame}[fragile]{Dangling View}
    \begin{overlayarea}{\linewidth}{6cm}
        \begin{minted}{c++}
std::vector<Pokemon> get_pokemon() {...};
auto hat_typ_feuer = [](const Pokemon &p) {
    return p.primaer_typ == "Feuer"
        || p.sekundaer_typ = "Feuer";
};
auto get_feuer_pokemon(const std::vector<Pokemon>& p) {
    return p | views::filter(hat_typ_feuer);
}
    
auto view = get_feuer_pokemon(get_pokemon())
for(const Pokemon& p : view)
    std::cout << p << std::endl;
        \end{minted}
        \begin{onlyenv}<2->
            \begin{minted}{c++}
//segfault
            \end{minted}
        \end{onlyenv}
    \end{overlayarea}
\end{frame}

%wann dangling view sinnvoll?

\begin{frame}[fragile]{Dangling View}
    \begin{minted}{c++}
std::string s{" Das Ist Ein Langer String. "};
auto s_view(const std::string& s) {
    [...]
    return std::string_view{s};
}
std::string_view = s_view(s);
auto view = sv
            | filter([](char c) { return c!= ' '; });
for (char c : view) {
    std::cout << c;
} //"DasIstEinLangerString"
    \end{minted}
\end{frame}

\begin{frame}[fragile]{Dangling View}
    \begin{minted}{c++}
std::string s{" Das Ist Ein Langer String. "};
auto s_view(const std::string& s) {
    [...]
    return std::string_view{s};
}
auto view = s_view(s)
            | filter([](char c) { return c!= ' '; });
for (char c : view) {
    std::cout << c;
} //"DasIstEinLangerString"
    \end{minted}
\end{frame}

\begin{frame}{\mintinline{c++}{std::ranges::viewable_range}}
    \begin{center}
        \textbf{Was kann man einem Range adaptor alles übergeben?}
    \end{center}

    \vspace{2.5em}

    \onslide<2->{\textbf{Alles was eine \mintinline{c++}{std::ranges::viewable_range} ist!}}

    \begin{itemize}
        \item<3-> lvalue reference (alles mit einem Namen)
        \item<4-> \mintinline{c++}{std::borrowed_range}
            \begin{itemize}
                \item<5-> Alle views
                \item<6-> Typen mit \mintinline{c++}{std::enable_borrowed_range<T>} true
                    \begin{itemize}
                        \item<7-> \mintinline{c++}{std::span, std::string_view}
                    \end{itemize}
            \end{itemize}
    \end{itemize}
\end{frame}

\begin{frame}[fragile]{Performance}
    \begin{minted}{c++}
std::vector<long long> eager_transform(
    const std::vector<long long> &vec) {
    std::vector<long long> result;
    for (long long i : vec) {
        if (i % 2 == 0){
            result.push_back(i * i);
        }
    }
    return result;
}
    \end{minted}
\end{frame}

\begin{frame}[fragile]{Performance}
    \begin{minted}{c++}
auto lazy_transform(
    const std::vector<long long> &vec) {
        auto is_even = [](long long i){
            return i % 2 == 0;
        };
        auto square = [](long long i){
            return i * i;
        };
        return vec | filter(is_even) | transform(square);
    }
    \end{minted}
\end{frame}

\begin{frame}[fragile]{Performance}
    \begin{minted}{c++}
long long do_something(auto &vec, int n) {
    long long sum = 0;
    for (auto i : vec){
        sum += i; n--;
        if (n < 0)
            break;
    }
    return sum;
}
    \end{minted}
\end{frame}

\begin{frame}[fragile]{Performance}
    \begin{minted}{c++}
int main() {
    std::vector<int> v = get_vektor(100000);
    auto v1 = eager_transform(v); // 3018µs
    auto v2 = layz_transform(v); // 0µs
    do_something(v1, 100000); // 425µs
    do_something(v2, 100000); // 8293µs
    do_something(v1, 1000); // 114µs
    do_something(v2, 1000); // 1352µs
}
    \end{minted}
\end{frame}

\begin{frame}[fragile]{Eigene View}
    \begin{minted}{c++}
template <std::ranges::input_range R>
        requires std::ranges::viewable_range<R>
    class view_convert : public std::ranges::view_base {
    private:
        R r;
    
        struct iterator_type : std::ranges::iterator_t<R> {
    \end{minted}
\end{frame}

\begin{frame}[fragile, allowframebreaks]{Eigene View - Iterator}
    \begin{minted}{c++}
struct iterator_type : std::ranges::iterator_t<R> {
    using base = std::ranges::iterator_t<R>;
    iterator_type() = default;
    iterator_type(base const &b) : base{b} {}

    iterator_type &operator++() {
        ++static_cast<base &>(*this);
        return (*this);
    }



    int operator*() const {
        auto original = *static_cast<base>(*this);
        return original*10;
    }

    int operator[](size_t n) const
    requires std::ranges::random_access_range<R> {
        auto original = (static_cast<base>(*this))[n];
        return original*10;
    }

};
    \end{minted}
\end{frame}

\begin{frame}[fragile, allowframebreaks]{Eigene View}
    \begin{minted}{c++}
public:
    view_convert(R &&r) : r{std::forward<R>(r)} {}

    iterator_type begin() const {
        return std::begin(r);
    }

    iterator_type end() const {
        return std::end(r);
    }


    int operator[](size_t n) const
    requires std::ranges::random_access_range<R>{
        return begin()[n];
    }
};
    \end{minted}
\end{frame}

\begin{frame}[fragile, allowframebreaks]{Eigene View - Adaptor}
    \begin{minted}{c++}
template <typename R>
view_convert(R &&) -> view_convert<R>;

struct convert_view_fn
{
    template <typename R>
    auto operator()(R &&r) const
    {
        return view_convert{std::forward<R>(r)};
    }


    template <typename R>
    friend auto operator|(R &&r, convert_view_fn const &)
    {
        return view_convert{std::forward<R>(r)};
    }
};

namespace view
{
    convert_view_fn convert;
}
    \end{minted}
\end{frame}

\begin{frame}[fragile, allowframebreaks]{Eigene View - Beispiel}
    \begin{minted}{c++}
std::vector<int> vec{1, 3, 5, 7, 9};
auto view = vec | view::convert;
std::cout << view[2] << std::endl; //50
for (auto i : view) {
    std::cout << i << ", ";
}
std::cout << std::endl; // 10, 30, 50, 70, 90,
    \end{minted}
\end{frame}