\subsubsection{Projection}

\begin{frame}{Projection}
    A \emph{projection} is a unary callable which may ne passed to most algorithms.

    \vspace{2.5em}

    Projections modify the view of the data that the algorithms sees.
\end{frame}

\begin{frame}[fragile]{Projection - Beispiel}
    \begin{minted}{cpp}
struct Pokemon {
    int id;
    std::string name;
    ...
}
struct PokedexEintrag {
    int pokemon_id;
    std::string beschreibung;
}

std::vector<Pokemon> pokemon;
std::vector<PokedexEintrag> pokedex;
    \end{minted}
\end{frame}

\begin{frame}{Projection - Beispiel}
    \textbf{Aufgabe:}

    Finde heraus, ob jedes Pokémon einen PokedexEintrag hat.
\end{frame}

\begin{frame}{Projection - Beispiel}
    \begin{center}
        \textbf{Erster Ansatz:}
    \end{center}
    \begin{enumerate}
        \item<1-> Sortiere alle Pokemon (nach \texttt{id})
        \item<2-> Sortiere alle PokedexEinträge (nach \texttt{pokemon\_id})
        \item<3-> Gehe von oben alle Einträge durch.
            Sobald die eine Liste die andere \enquote{überholt}, gib \texttt{false} zurück.
        \item<4-> Wenn wir am Ende der Listen sind, gib \texttt{true} zurück
    \end{enumerate}
\end{frame}

\begin{frame}[fragile]{Projection - Beispiel}
    \begin{minted}{cpp}
std::sort(pokemon.begin(), pokemon.end(),
    [] (const Pokemon& p1, const Pokemon& p2) {
        return p1.id <= p2.id;
    });

std::sort(pokedex.begin(), pokedex.end(),
    [] (const PokedexEintrag& e1,
        const PokedexEintrag& e2) {
        return e1.pokemon_id <= e2.pokemon_id;
    });
    \end{minted}
\end{frame}

\begin{frame}[fragile]{Projection - Beispiel}
    \begin{minted}{cpp}
std::equal(
    pokemon.begin(), pokemon.end(),
    pokedex.begin(), pokedex.end(),
    [] (const Pokemon& p, const PokedexEintrag& e) {
        return p.id == e.pokemon_id;
    });
    \end{minted}
\end{frame}

\begin{frame}{Projection - Beispiel}
    \begin{center}
        \textbf{Zweiter Ansatz:}
    \end{center}

    \begin{itemize}
        \item Erweitere Iteratoren durch Ranges

              \begin{itemize}
                  \item Ersetze \texttt{.begin()} bzw. \texttt{.end()}
              \end{itemize}
    \end{itemize}
\end{frame}

\begin{frame}[fragile]{Projection - Beispiel}
    \begin{minted}{cpp}
std::ranges::sort(pokemon,
    [] (const Pokemon& p1, const Pokemon& p2) {
        return p1.id <= p2.id;
    });

std::ranges::sort(pokedex,
    [] (const PokedexEintrag& e1,
        const PokedexEintrag& e2) {
        return e1.pokemon_id <= e2.pokemon_id;
    });
    \end{minted}
\end{frame}

\begin{frame}[fragile]{Projection - Beispiel}
    \begin{minted}{cpp}
std::ranges::equal(pokemon, pokedex,
    [] (const Pokemon& p, const PokedexEintrag& e) {
        return p.id == e.pokemon_id;
    });
    \end{minted}
\end{frame}

\begin{frame}{Projection - Beispiel}
    \begin{center}
        \textbf{Dritter Ansatz:}
    \end{center}

    \begin{itemize}
        \item Erweitere Ranges durch Projection

              \begin{itemize}
                  \item Reduziere jedes Pokemon auf einen \texttt{int}
                  \item Reduziere analog jeden PokedexEintrag auf einen \texttt{int}
                  \item Überlasse dem Compiler den Umgang mit diesen primitiven Datentypen
              \end{itemize}
    \end{itemize}
\end{frame}

\begin{frame}[fragile]{Projection - Beispiel}
    \begin{minted}{cpp}
std::ranges::sort(pokemon, std::ranges::less{},
    [] (const Pokemon& p) { return p.id; });

std::ranges::sort(pokedex, std::ranges::less{},
    [] (const PokedexEintrag& e) { return e.pokemon_id; });

std::ranges::equal(pokemon, pokedex,
    std::ranges::equal_to{},
    [] (const Pokemon& p) { return p.id; }
    [] (const PokedexEintrag& e) { return e.pokemon_id; }
);
    \end{minted}
\end{frame}

\begin{frame}{Projection - Beispiel}
    \begin{center}
        \textbf{Finaler Ansatz:}
    \end{center}

    \begin{itemize}
        \item Ersetze \enquote{getter}-Lambdas durch die direkten Attribute
    \end{itemize}
\end{frame}

\begin{frame}[fragile]{Projection - Beispiel}
    \begin{minted}{cpp}
std::ranges::sort(pokemon, std::ranges::less{},
    &Pokemon::id);

std::ranges::sort(pokedex, std::ranges::less{},
    &PokedexEintrag::pokemon_id);

std::ranges::equal(pokemon, pokedex,
    std::ranges::equal_to{},
    &Pokemon::id, &PokedexEintrag::pokemon_id);
    \end{minted}
\end{frame}

\begin{frame}{Projection - Beispiel}
    \begin{center}
        \SixStar \ \textbf{Finaler Ansatz:} \SixStar
    \end{center}
\end{frame}

\begin{frame}[fragile]{Projection - Beispiel}
    \begin{minted}{cpp}
std::ranges::sort(pokemon, {},
    &Pokemon::id);

std::ranges::sort(pokedex, {},
    &PokedexEintrag::pokemon_id);

std::ranges::equal(pokemon, pokedex, {},
    &Pokemon::id, &PokedexEintrag::pokemon_id);
    \end{minted}
\end{frame}

\begin{frame}[fragile]{Projection - Beispiel}
    Zusammengefasst:

    \begin{minted}{cpp}
// stl
std::sort(pokemon.begin(), pokemon.end(),
    [] (const Pokemon& p1, const Pokemon& p2) {
        return p1.id <= p2.id;
    });

// ranges
std::ranges::sort(pokemon, {}, &Pokemon::id);
    \end{minted}
\end{frame}