\begin{frame}[fragile]{(Compiler-) Geschwindigkeitsoptimierung}
    \begin{minted}{cpp}
std::vector<int> v{ 3, 1, 4, 1, 5, ... };

std::find(
    v.begin(), // iterator -> v[0]
    v.end(),   // iterator -> nullptr
    5          // ist immer im vector enthalten
);
    \end{minted}
\end{frame}

\begin{frame}[fragile]{(Compiler-) Geschwindigkeitsoptimierung}
    \begin{minted}{cpp}
std::vector<int>::iterator find(
    std::vector<int>::iterator first,
    std::vector<int>::iterator last,
    const int&                 val
) {
    for (; first != last; ++first)
        if (*first == val) break;
    
    return first;
}
    \end{minted}
\end{frame}

\begin{frame}[fragile]{(Compiler-) Geschwindigkeitsoptimierung}
    \begin{minted}{cpp}
std::vector<int> v{ 3, 1, 4, 1, 5, ... };

std::find(
    v.begin(), // iterator -> v[0]
    ranges::unreachable_sentinal, // == it -> false
    5          // ist immer im vector enthalten
);
    \end{minted}
\end{frame}

\begin{frame}{(Compiler-) Geschwindigkeitsoptimierung}
    \begin{center}
        A sentinel is some type that is \texttt{equality\_comparable\_with} its corresponding iterator, which denotes the end of the range.

        \vspace{2.5em}

        Für diese Beispiel benötigen wir den \texttt{unreachable\_sentinal}.
    \end{center}
\end{frame}

\begin{frame}[fragile]{(Compiler-) Geschwindigkeitsoptimierung}
    \begin{minted}{cpp}
std::vector<int>::iterator find(
    std::vector<int>::iterator     first,
    ranges::unreachable_sentinal_t last,
    const int&                     val
) {
    // first == last -> false
    for (; first != last; ++first)
        if (*first == val) break;
    
    return first;
}
    \end{minted}
\end{frame}

\begin{frame}[fragile]{(Compiler-) Geschwindigkeitsoptimierung}
    \begin{minted}{cpp}
std::vector<int>::iterator find(
    std::vector<int>::iterator     first,
    ranges::unreachable_sentinal_t last,
    const int&                     val
) {
    // first == last -> false
    for (; true; ++first)
        if (*first == val) break;
    
    return first;
}
    \end{minted}
\end{frame}

\begin{frame}[fragile]{(Compiler-) Geschwindigkeitsoptimierung}
    Mit dem bekannten Iterator \texttt{.end()} wird jedes Element des Vectors 2 mal \enquote{angefasst}:
    \begin{itemize}
        \item \mintinline{cpp}{first != last}
        \item \mintinline{cpp}{*first == val}
    \end{itemize}

    \vspace{2.5em}

    Durch Sentinals verringern wir diese Zugriffe um die Hälfte, haben jedoch Probleme, wenn \texttt{val} nicht im Vector enthalten ist.
\end{frame}